\documentclass[12pt]{article}
\usepackage{graphics}
\usepackage{graphicx}

\title{Reflection and Assessment Plan}
\author{Ali Asghar Yousuf - ay06993}
\date{\today}

\begin{document}
\maketitle

\section*{Introduction}
This document will highlight the goals I set for myself at the start of the course project. I will discuss whether the goals were successfuly and if they were not successful what were the reasons behind it.

\section*{Goals}
\subsection*{Successfully complete the project}
The first goal was to successfully complete the project and get satisfying
results. I planned on measuring this through the number of hours I put in
studying the theoretical concepts of mobile robotics. Something new that I did
for this course was that I referred the suggested course books alot. I found
the books extremely useful for preparation of quizzes, attempting homework
questions, and even trying to find something useful related to the project.

While I was unable to put in as many hours as I initially wanted and that a
mobile robotics project deserves, I believe that the number of hours in this
project start from the time where I study theoretical concepts. In addition,
the iterative process of trying to implementing a floor cleaning robot was
interesting on its own. There are a lot of ways that a floor cleaning robot
could be implemented but focusing on the coverage problem and trying to
implementing obstacle avoidance with it is one of the best ways I found to
implement a floor cleaning robot.

The evidence of this goal being achieved is the final paper that we submitted.
There are obvious improvements that can be made but getting the results from
where we started in the course with almost knowing nothing about robotics is a
big step on its own and I would give myself A for Affort (yes the spelling of
effort is knowingly done).

\subsection*{Attain Technical Skills}
At the start of the course project I wanted to master the following;

\begin{itemize}
    \item \texttt{MATLAB}
    \item \texttt{Simulink}
    \item \texttt{ROS}
    \item \texttt{Gazebo}
\end{itemize}

The assessment plan for this goal was to assess my progress by updating a
GitHub repository and a log of all the details and struggles I went through. I
have been updating a GitHub repository for the project, which is mainly divided
into two main folders, one being for the documents and the other containing all
the code files. While I did not explicitly keep a log of all the progress I
have been making, I believe my GitHub commits and commit messages.

To further practise and learn more about MATLAB and Simulink, I was running the
default examples and tutorials provided by Mathworks, for example mlx files for
\textit{MonteCarloLocalization} and \textit{Occupancy grid mapping}. Running
different Matlab examples really helped me get a hang of how Matlab works and
the little things for example how to print the coordinates saved in a
\textbf{.mat} file. At times, during the implementation of the project and
homeworks, I also went through a couple of ROS tutorials.

While I would say I now know 30\% of ROS and Matlab, and that may not sound
like a lot, but I feel like this is a significant jump from where I started at
the beginning of the course. Knowing almost next to nothing about the Matlab
and ROS, I now feel like with a bit of more self-learning and dedication I can
take on entire projects myself.

\subsection*{Implement Path Coverage Algorithms}
Implement path coverage algorithms is one of the fundamental goal for this
project which is also directly linked to assessing the success of the project.
I want to assess my goal based on the number of hours I spend implementing the
algorithms and the quality of my results. I tried implementing occupancy grid
mapping and spiral random movement with Vector field Historgram algorithm. VFH
algorithm was also used in homework 3 and 4, so I know a little about it and
how I can successfully use it. We ended up deciding on implementing spiral random movement with the VFH algorithm because in that manner it gives us a better chance of coverage theoretically if we run it for some x amount of time. 
In addition, spiral random movement is found to cover more area when compared to random walk and snake movement. I will give myself a B for Best in this goal. 

\end{document}